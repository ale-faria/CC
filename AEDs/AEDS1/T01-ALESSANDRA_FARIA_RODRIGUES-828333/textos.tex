%%%%%%%%%%%%%%%%%%%%%%%%%%%%%%%%%%%%%%%%%%%%%%%%%%%%%%%%%%%%%%%%%%%%%%%%%%%%%%%%%%%%%%%%%%%%%%%%%%%%%%%
%%%%%%%%%%%%%% Template de Artigo Adaptado para Trabalho de Diplomação do ICEI %%%%%%%%%%%%%%%%%%%%%%%%
%% codificação UTF-8 - Abntex - Latex -  							     %%
%% Autor:    Fábio Leandro Rodrigues Cordeiro  (fabioleandro@pucminas.br)                            %% 
%% Co-autores: Prof. João Paulo Domingos Silva, Harison da Silva e Anderson Carvalho		     %%
%% Revisores normas NBR (Padrão PUC Minas): Helenice Rego Cunha e Prof. Theldo Cruz                  %%
%% Versão: 1.1     18 de dezembro 2015                                                               %%
%%%%%%%%%%%%%%%%%%%%%%%%%%%%%%%%%%%%%%%%%%%%%%%%%%%%%%%%%%%%%%%%%%%%%%%%%%%%%%%%%%%%%%%%%%%%%%%%%%%%%%%
\section{\esp Introdução}

A pesquisa científica sempre começa com uma pergunta, a partir dessa pergunta é possível criar respostas, essas possíveis respostas são chamadas de hipótese. Uma hipótese científica é uma suposição especulativa que é aceita de forma provisória e serve como uma explicação inicial para um fenômeno observado. Ela é formulada com base em dados existentes e conhecimentos prévios, e pode ser testada através da observação e experimentação. Dentro desse contexto, vale ressaltar que nem toda pesquisa científica apresenta a necessidade de se criar hipóteses, sendo assim, é possível analisar dois tipos distintos de pesquisa: a pesquisa com hipótese e a pesquisa sem hipótese.

\section{\esp Desenvolvimento}

Em primeira análise, evidencia-se a pesquisa onde não há a necessidade de hipótese, também chamada de pesquisa descritiva. Nesse tipo de pesquisa a metodologia é determinada pela pergunta, dessa forma, o foco geralmente está na coleta de dados e na observação, sem uma previsão específica sobre o que será encontrado. Nesse método o pesquisador busca entender melhor um determinado objeto de estudo ou fenômeno, sem ter uma direção clara no início da pesquisa.
	\par Já a pesquisa com hipótese testa a relação entre duas ou mais variáveis e pode ser subdividida em dois tipos: relação por associação e relação por interferência. 
\par A pesquisa por associação diz sobre correlações entre variáveis que se alteram proporcionalmente, mas uma não interfere na outra. Um exemplo de associação é dizer que “quanto mais filmes com o Nicolas Cage são lançados por ano, menos pessoas morrem em acidentes de helicóptero”, ou seja, são dois dados estatísticos que estão sim correlacionados, porém, um não influencia no outro.
\par Por outro lado, a pesquisa por interferência diz sobre variáveis onde uma tem a capacidade de influenciar a outra. Nesse tipo de relação há um mecanismo responsável por causar certo efeito. Como exemplo, o consumo de cigarro sendo responsável por causar infarto do miocárdio, nesse caso, a ação está diretamente influenciando no efeito. 



\section{\esp Conclusão}
Diante dos tipos de pesquisa científica abordados, torna-se evidente a importância da formulação de hipóteses como ponto de partida em muitos estudos. A hipótese direciona a investigação e também fornece uma estrutura base para testar e validar ideias. No entanto, é igualmente relevante reconhecer a importância da pesquisa sem hipótese, onde a exploração do objeto de estudo é conduzida de forma mais livre. Assim, ambas as abordagens contribuem de maneira distinta e significativa para o avanço do conhecimento científico.


% \subsection{\esp Trabalhos futuros}
% 
% Sugestões de estudos posteriores são ser adicionados subseção deste capítulo de conclusão.
