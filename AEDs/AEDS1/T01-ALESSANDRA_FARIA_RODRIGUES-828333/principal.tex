%%%%%%%%%%%%%%%%%%%%%%%%%%%%%%%%%%%%%%%%%%%%%%%%%%%%%%%%%%%%%%%%%%%%%%%%%%%%%%%%%%%%%%%%%%%%%%%%%%%%%%%
%%%%%%%%%%%%%% Template de Artigo Adaptado para Trabalho de Diplomação do ICEI %%%%%%%%%%%%%%%%%%%%%%%%
%% codificação UTF-8 - Abntex - Latex -  							     %%
%% Autor:    Fábio Leandro Rodrigues Cordeiro  (fabioleandro@pucminas.br)                            %% 
%% Co-autor: Prof. João Paulo Domingos Silva, Harison da Silva e Anderson Carvalho                   %%
%% Revisores normas NBR (Padrão PUC Minas): Helenice Rego Cunha e Prof. Theldo Cruz                  %%
%% Versão: 1.1     18 de dezembro 2015                     	                                     %%
%%%%%%%%%%%%%%%%%%%%%%%%%%%%%%%%%%%%%%%%%%%%%%%%%%%%%%%%%%%%%%%%%%%%%%%%%%%%%%%%%%%%%%%%%%%%%%%%%%%%%%%


\documentclass[a4paper,12pt,Times]{article}
\usepackage{abakos}  %pacote com padrão da Abakos baseado no padrão da PUC

%%%%%%%%%%%%%%%%%%%%%%%%%%%
%Capa da revista
%%%%%%%%%%%%%%%%%%%%%%%%%%

%\setcounter{page}{80} %iniciar contador de pagina de valor especificado
\newcommand{\monog}{Síntese: os três tipos lógicos de pesquisa}
%\newcommand{\monogES}{Article template Institute of Mathematical Sciences and Informatics}
\newcommand{\tipo}{Artigo }  % Especificar a seção tipo do trabalho: Artigo, Resumo, Tese, Dociê etc
\newcommand{\origem}{Brasil }
\newcommand{\editorial}{Belo Horizonte, p. 01-4, mar. 2024}  % p. xx-xx – páginas inicial-final do artigo
\newcommand{\lcc}{\scriptsize{Licença Creative Commons Attribution-NonCommercial-NoDerivs 3.0 Unported}}

%%%%%%%%%%%%%%%%%INFORMAÇÕES SOBRE AUTOR PRINCIPAL %%%%%%%%%%%%%%%%%%%%%%%%%%%%%%%
\newcommand{\AutorA}{Alessandra Faria Rodrigues}
\newcommand{\funcaoA}{}
\newcommand{\emailA}{alessandra.rodrigues.1416939@sga.pucminas.br}
\newcommand{\cursA}{Aluna da Graduação em Ciência da Computação}

%\newcommand{\AutorB}{Fábio Leandro Rodrigues Cordeiro}
%\newcommand{\funcaoB}{}
%\newcommand{\emailB}{fabioleandro@pucminas.br}
%\newcommand{\cursB}{Professor do Programa de Graduação em Ciência da Computação}
% 
% Definir macros para o nome da Instituição, da Faculdade, etc.
\newcommand{\univ}{Pontifícia Universidade Católica de Minas Gerais}

\newcommand{\keyword}[1]{\textsf{#1}}

\begin{document}
% %%%%%%%%%%%%%%%%%%%%%%%%%%%%%%%%%%
% %% Pagina de titulo
% %%%%%%%%%%%%%%%%%%%%%%%%%%%%%%%%%%

\begin{center}
\includegraphics[scale=0.2]{figuras/brasao.jpg} \\
PONTIFÍCIA UNIVERSIDADE CATÓLICA DE MINAS GERAIS \\
Instituto de Ciências Exatas e de Informática

% \vspace{1.0cm}

\end{center}

 \vspace{0cm} {
 \singlespacing \Large{\monog \symbolfootnote[1]{} \\ }
 % \normalsize{\monogES}
 }

\vspace{1.0cm}

\begin{flushright}
\singlespacing 
\normalsize{\AutorA \footnote{\funcaoA \cursA, \origem -- \emailA . }} \\
%\normalsize{\AutorB \footnote{\funcaoB \cursB, \origem -- \emailB . }} \\
%\normalsize{\AutorC \footnote{\funcaoC \cursC, \origem -- \emailC . }} \\
%\normalsize{\AutorD \footnote{\funcaD \cursD, \origem -- \emailD . }} \\
%deixar com o valor `0` e usar o '*' no inicio da frase
% \symbolfootnote[0]{Artigo recebido em 10 de julho de 1983 e aprovado em 29 de maio 2012}
\end{flushright}
\thispagestyle{empty}

\vspace{1.0cm}

\begin{abstract}
\noindent
Esta síntese abordará os tipos lógicos de pesquisa, sendo eles: pesquisa sem hipótese (descritiva), pesquisa com hipótese com relação de associação e pesquisa com hipótese com relação de interferência. Apontando as características e peculiaridades de cada tipo, e ressaltando a importância de cada tipo de pesquisa. Esta síntese é baseada no vídeo de título “três tipos lógicos de pesquisa” do Dr. Gilson Luiz Volpato, que está disponível para visualização no YouTube.
\\\textbf{\keyword{Palavras-chave: }} pesquisa científica; hipótese.

\end{abstract}

%%%%%%%%%%%%%%%%%%%%%%%%%%%%%%%%%%%%%%%%%%%%%%%%%%%%%%%%%
% \newpage    %%%% CASO QUEIRA QUE O RESUMO FIQUE EM UMA PAGINA E O ABSTRACT EM OUTRA
%\selectlanguage{english}
%\begin{abstract}
%\noindent
%The abstract should contain at least one hundred and fifty words in accordance with the standards of ABNT standard.
%This present article will address the main features of the web programming languages, that are used currently, 
%comparing its features as to indicate to indicate the better use of determined language. 
%The linguage will be divided according with four major caracteristics: Interpreted, compiled, server-side and client-side.
%This present article will address the main features of the web programming languages.
%The abstract should contain at least one hundred and fifty words in accordance with the standards of ABNT standard.
%The linguage will be divided according with four major caracteristics: Interpreted, compiled, server-side and client-side.
%This present article will address the main features of the web programming languages.
%The abstract should contain at least one hundred and fifty words in accordance with the standards of ABNT standard.
%\\\textbf{\keyword{Keywords: }} Template. %\LaTeX. Abakos. Periodics.
%\end{abstract}

\selectlanguage{brazilian}
 \onehalfspace  % espaçamento 1.5 entre linhas
 \setlength{\parindent}{1.25cm}

%%%%%%%%%%%%%%%%%%%%%%%%%%%%%%%%%%%%%%%%%%%%%%%%%
%% INICIO DO TEXTO
%%%%%%%%%%%%%%%%%%%%%%%%%%%%%%%%%%%%%%%%%%%%%%%%%

%%%%%%%%%%%%%%%%%%%%%%%%%%%%%%%%%%%%%%%%%%%%%%%%%%%%%%%%%%%%%%%%%%%%%%%%%%%%%%%%%%%%%%%%%%%%%%%%%%%%%%%
%%%%%%%%%%%%%% Template de Artigo Adaptado para Trabalho de Diplomação do ICEI %%%%%%%%%%%%%%%%%%%%%%%%
%% codificação UTF-8 - Abntex - Latex -  							     %%
%% Autor:    Fábio Leandro Rodrigues Cordeiro  (fabioleandro@pucminas.br)                            %% 
%% Co-autores: Prof. João Paulo Domingos Silva, Harison da Silva e Anderson Carvalho		     %%
%% Revisores normas NBR (Padrão PUC Minas): Helenice Rego Cunha e Prof. Theldo Cruz                  %%
%% Versão: 1.1     18 de dezembro 2015                                                               %%
%%%%%%%%%%%%%%%%%%%%%%%%%%%%%%%%%%%%%%%%%%%%%%%%%%%%%%%%%%%%%%%%%%%%%%%%%%%%%%%%%%%%%%%%%%%%%%%%%%%%%%%
\section{\esp Introdução}

A pesquisa científica sempre começa com uma pergunta, a partir dessa pergunta é possível criar respostas, essas possíveis respostas são chamadas de hipótese. Uma hipótese científica é uma suposição especulativa que é aceita de forma provisória e serve como uma explicação inicial para um fenômeno observado. Ela é formulada com base em dados existentes e conhecimentos prévios, e pode ser testada através da observação e experimentação. Dentro desse contexto, vale ressaltar que nem toda pesquisa científica apresenta a necessidade de se criar hipóteses, sendo assim, é possível analisar dois tipos distintos de pesquisa: a pesquisa com hipótese e a pesquisa sem hipótese.

\section{\esp Desenvolvimento}

Em primeira análise, evidencia-se a pesquisa onde não há a necessidade de hipótese, também chamada de pesquisa descritiva. Nesse tipo de pesquisa a metodologia é determinada pela pergunta, dessa forma, o foco geralmente está na coleta de dados e na observação, sem uma previsão específica sobre o que será encontrado. Nesse método o pesquisador busca entender melhor um determinado objeto de estudo ou fenômeno, sem ter uma direção clara no início da pesquisa.
	\par Já a pesquisa com hipótese testa a relação entre duas ou mais variáveis e pode ser subdividida em dois tipos: relação por associação e relação por interferência. 
\par A pesquisa por associação diz sobre correlações entre variáveis que se alteram proporcionalmente, mas uma não interfere na outra. Um exemplo de associação é dizer que “quanto mais filmes com o Nicolas Cage são lançados por ano, menos pessoas morrem em acidentes de helicóptero”, ou seja, são dois dados estatísticos que estão sim correlacionados, porém, um não influencia no outro.
\par Por outro lado, a pesquisa por interferência diz sobre variáveis onde uma tem a capacidade de influenciar a outra. Nesse tipo de relação há um mecanismo responsável por causar certo efeito. Como exemplo, o consumo de cigarro sendo responsável por causar infarto do miocárdio, nesse caso, a ação está diretamente influenciando no efeito. 



\section{\esp Conclusão}
Diante dos tipos de pesquisa científica abordados, torna-se evidente a importância da formulação de hipóteses como ponto de partida em muitos estudos. A hipótese direciona a investigação e também fornece uma estrutura base para testar e validar ideias. No entanto, é igualmente relevante reconhecer a importância da pesquisa sem hipótese, onde a exploração do objeto de estudo é conduzida de forma mais livre. Assim, ambas as abordagens contribuem de maneira distinta e significativa para o avanço do conhecimento científico.


% \subsection{\esp Trabalhos futuros}
% 
% Sugestões de estudos posteriores são ser adicionados subseção deste capítulo de conclusão.


%%%%%%%%%%%%%%%%%%%%%%%%%%%%%%%%%%%
%% FIM DO TEXTO
%%%%%%%%%%%%%%%%%%%%%%%%%%%%%%%%%%%

% \selectlanguage{brazil}
%%%%%%%%%%%%%%%%%%%%%%%%%%%%%%%%%%%
%% Inicio bibliografia
%%%%%%%%%%%%%%%%%%%%%%%%%%%%%%%%%%%

 \newpage
\singlespace{
\renewcommand\refname{REFERÊNCIAS}
\bibliographystyle{abntex2-alf}
\bibliography{bibliografia}


}
 VOLPATO, Gilson. AULA 20 de 42 - TRÊS TIPOS LÓGICOS DE PESQUISA. YouTube, 15 de março de 2012. Disponível em: https://youtu.be/XoTQo7fUf0s?si=AtJ5rrJLWg0w5vIS. Acesso em: 28 de março de 2024.
\end{document}


